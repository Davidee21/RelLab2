\documentclass{article}
\usepackage[italian]{babel}
\usepackage[utf8]{inputenc}
\usepackage{fancyhdr}
\usepackage{tikz}
\usepackage{amsmath}
\usepackage{amssymb}
\usepackage{amsthm}
\usepackage{amsfonts}
\usepackage{color}
\usepackage[margin=2cm]{geometry}
\usepackage{titlesec}

\titleformat{\paragraph}
  {\normalfont\normalsize\bfseries}{\theparagraph}{1em}{}
\titlespacing*{\paragraph}
  {0pt}{3.25ex plus 1ex minus .2ex}{1.5ex plus .2ex}

\title{Analisi di un circuito RLC serie in regime sinusoidale}
\date{20/05/2022}
\author{Bertasi Leonardo, Perniola Davide}
\begin{document}
\maketitle
\section{Abstract} 


\section{Introduzione} 
Un circuito RLC serie consiste in una resistenza, una induttanza e un condensatore posti in serie. Applicando ai capi del ciruito una differenza di potenziale sinusoidale(aggiungere) ci si aspetta di osservare un preciso andamento, anch'esso sinusoidale, in ognuno degli elementi circuitali. La differenza di potenziale ai capi della resistenza segue la relazione
$$
$$





\end{document}

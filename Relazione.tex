\documentclass{article}
\usepackage[italian]{babel}
\usepackage[utf8]{inputenc}
\usepackage{fancyhdr}
\usepackage{tikz}
\usepackage{amsmath}
\usepackage{amssymb}
\usepackage{amsthm}
\usepackage{amsfonts}
\usepackage{color}
\usepackage{circuitikz}
\usepackage[margin=2cm]{geometry}
\usepackage{titlesec}

\titleformat{\paragraph}
  {\normalfont\normalsize\bfseries}{\theparagraph}{1em}{}
\titlespacing*{\paragraph}
  {0pt}{3.25ex plus 1ex minus .2ex}{1.5ex plus .2ex}

\title{Analisi di un circuito RLC serie in regime sinusoidale}
\date{20/05/2022}
\author{Bertasi Leonardo, Perniola Davide}
\begin{document}
\maketitle
\section{Abstract} 


\section{Introduzione} 
Un circuito RLC serie consiste in una resistenza, una induttanza e un condensatore posti in serie. Applicando ai capi del ciruito una differenza di potenziale sinusoidale $V_{0}\cos{wt} $ ci si aspetta di osservare un preciso andamento, anch'esso sinusoidale,
 ai capi di ognuno degli elementi circuitali. L'unica corrente che scorre nel circuito segue la relazione(si veda appendice) 
\begin{equation}
  i(t)=\frac{V_{0}}{\sqrt{R^2+(wL-\frac{1}{wC})^2}}\cos{[wt+(\arctan{\frac{1-w^2LC}{wRC}})]}
\end{equation}
Utilizzando la (1) si possono scrivere gli andamenti teorici della ddp ai capi della resistenza
\begin{equation}
 V_{R}(t)= \frac{V_{0}R}{\sqrt{R^2+(wL-\frac{1}{wC})^2}}\cos{[wt+(\arctan{\frac{1-w^2LC}{wRC}})]}
\end{equation}

dell'induttanza
\begin{equation}
  V_{L}(t)=\frac{V_{0}wL}{\sqrt{R^2+(wL-\frac{1}{wC})^2}}\cos{[wt+(\arctan{\frac{1-w^2LC}{wRC}})+\frac{\pi}{2}]}
\end{equation}
e del condensatore
\begin{equation}
  V_{C}(t)=\frac{(\frac{V_{0}}{wC})}{\sqrt{R^2+(wL-\frac{1}{wC})^2}}\cos{[wt+(\arctan{\frac{1-w^2LC}{wRC}})-\frac{\pi}{2}]}
\end{equation}
Parlare di obiettivi e freq risonanza

\section{Apparato sperimentale e svolgimento}
\begin{figure}[h!]
  \begin{center}
    \begin{circuitikz}
      \draw (0,0)
      to[voltage source=$V_S$] (0,2) % The voltage source
      to[R=$R_1$] (4,2) % The resistor
      to[short] (4,2)
      to[L=$L$] (4,0)
      to[C=$C$] (0,0)
      to[short] (0,0);

    \end{circuitikz}
    \caption{\textit{Schema del circuito realizzato.}}
  \end{center}
\end{figure}

\end{document}
